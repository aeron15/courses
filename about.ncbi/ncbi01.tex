  \documentclass{beamer}
\usepackage[utf8]{inputenc}
\usepackage{hyperref}
\usepackage{graphicx}
\usepackage{listings}
\lstset{frame=single,backgroundcolor=\color{lightgray}}
\usetheme{Warsaw}

\newcommand{\remoteimage}[3]{
\IfFileExists{#1}{}{\immediate\write18{curl -o "#1" "#2"}}
\begin{center}
\includegraphics[#3]{#1}
\end{center}
}

\newcommand{\graphviz}[3]{
\IfFileExists{#1}{}{\immediate\write18{echo #2 | dot -o"#1.png" -Tpng}}
\begin{center}
\includegraphics[#3]{#1.png}
\end{center}
}

\newcommand{\centeredtitle}[1]{
\begin{center}
    \Huge{\bf{#1}}
\end{center}
}

\newcommand{\hugeslide}[1]{
\begin{frame}
\centeredtitle{#1}
\end{frame}
}



\title{Advanced NCBI.\\\href{https://github.com/lindenb/courses}{https://github.com/lindenb/courses}}
\author{Pierre Lindenbaum\\\href{https://twitter.com/yokofakun}{@yokofakun}\\ \href{mailto:plindenbaum@yahoo.fr}{pierre.lindenbaum@univ-nantes.fr}\\ \url{http://plindenbaum.blogspot.com} }\institute{Institut du Thorax. Nantes. France}

\begin{document}



\begin{frame} 
\titlepage
\end{frame}

\begin{frame}
What we will cover:
\end{frame}

\begin{frame}
\remoteimage{jeter01.png}{http://www.ncbi.nlm.nih.gov/corehtml/pmc/pmcgifs/bookshelf/thumbs/th-helpeutils-lrg.png}{}
\end{frame}



\begin{frame} 
 The E-Utilities are a set of seven server-side programs that provide a stable interface to the search, retrieval, and linking functions of the Entrez system, using a fixed URL syntax. The output provided by the E-Utilities is in XML format, with the notable exception of the EFetch utility, which returns data in a variety of formats. The E-Utilities are designed to be called from within a computer program that can process their output. Calling an E-Utility from any of the common programming languages— including Perl, Python, and Java—is a simple matter of posting a URL.
 \end{frame}
 

%%%%%%%%%%%%%%%%%%%%%%%%%%%%%%%%%%%%%%%%%%%
%%
%% EINFO
%%
%%%%%%%%%%%%%%%%%%%%%%%%%%%%%%%%%%%%%%%%%%%

\hugeslide{EInfo}

\begin{frame}[fragile]
\frametitle{EInfo}
Provides a list of the names of all valid Entrez databases.\\
Provides statistics for a single database, including lists of indexing fields and available link names.
\end{frame}


\begin{frame}[fragile]
\frametitle{EInfo}
\small
\url{http://eutils.ncbi.nlm.nih.gov/entrez/eutils/einfo.fcgi}
\end{frame}



\begin{frame}[fragile]
\frametitle{EInfo}
\framesubtitle{XML Ouput}

\small
\begin{lstlisting}[language=xml]
<!DOCTYPE eInfoResult PUBLIC "-//NLM//DTD eInfoResult, 11 May 2002//EN" "http://www.ncbi.nlm.nih.gov/entrez/query/DTD/eInfo_020511.dtd">
<eInfoResult>
  <DbList>
    <DbName>pubmed</DbName>
    <DbName>protein</DbName>
    <DbName>nuccore</DbName>
    <DbName>nucleotide</DbName>
    <DbName>nucgss</DbName>
    <DbName>nucest</DbName>
    <DbName>structure</DbName>
    <DbName>genome</DbName>
    <DbName>assembly</DbName>
    <DbName>gcassembly</DbName>
    <DbName>genomeprj</DbName>
    <DbName>bioproject</DbName>
    <DbName>biosample</DbName>
    <DbName>biosystems</DbName>
    <DbName>blastdbinfo</DbName>
    <DbName>books</DbName>
    <DbName>cdd</DbName>
    <DbName>clinvar</DbName>
 (...)
\end{lstlisting}
\end{frame}


\begin{frame}[fragile]
\frametitle{EInfo}
Return statistics for Entrez database:\\
\small
\url{http://eutils.ncbi.nlm.nih.gov/entrez/eutils/einfo.fcgi?db=DbName}
\end{frame}


\begin{frame}[fragile]
\frametitle{EInfo}
\framesubtitle{Statistics for Pubmed}
\url{http://eutils.ncbi.nlm.nih.gov/entrez/eutils/einfo.fcgi?db=pubmed}
\begin{lstlisting}[language=xml,basicstyle=\tiny,breaklines=false]
<?xml version="1.0"?>
<!DOCTYPE eInfoResult PUBLIC "-//NLM//DTD eInfoResult, 11 May 2002//EN" "http://www.ncbi.nlm.nih.gov/entrez/query/DTD/eInfo_020511.dtd">
<eInfoResult>
  <DbInfo>
    <DbName>pubmed</DbName>
    <MenuName>PubMed</MenuName>
    <Description>PubMed bibliographic record</Description>
    <DbBuild>Build130805-2117m.4</DbBuild>
    <Count>22974581</Count>
    <LastUpdate>2013/08/06 08:33</LastUpdate>
    <FieldList>
      (...)
      <Field>
        <Name>UID</Name>
        <FullName>UID</FullName>
        <Description>Unique number assigned to publication</Description>
        <TermCount>0</TermCount>
        <IsDate>N</IsDate>
        <IsNumerical>Y</IsNumerical>
        <SingleToken>Y</SingleToken>
        <Hierarchy>N</Hierarchy>
        <IsHidden>Y</IsHidden>
      </Field>
      <Field>
(...)
\end{lstlisting}
\end{frame}

\begin{frame}[fragile]
\frametitle{EInfo}
\framesubtitle{Statistics for Pubmed}
\url{http://eutils.ncbi.nlm.nih.gov/entrez/eutils/einfo.fcgi?db=pubmed}
\begin{lstlisting}[language=xml,basicstyle=\tiny,breaklines=false]
(...)
      <Link>
        <Name>pubmed_taxonomy_entrez</Name>
        <Menu>Taxonomy via GenBank</Menu>
        <Description>Taxonomy records associated with the current articles through taxonomic information on related molecular database records (Nucleotide, Protein, Gene, SNP, Structu
re).</Description>
        <DbTo>taxonomy</DbTo>
      </Link>
      <Link>
        <Name>pubmed_unigene</Name>
        <Menu>UniGene Links</Menu>
        <Description>UniGene clusters of expressed sequences that are associated with the current articles through references on the clustered sequence records and related Gene record
s.</Description>
        <DbTo>unigene</DbTo>
      </Link>
      <Link>
        <Name>pubmed_unists</Name>
        <Menu>UniSTS Links</Menu>
        <Description>Genetic, physical, and sequence mapping reagents in the UniSTS database associated with the current articles through references on sequence tagged site (STS) subm
issions as well as automated searching of PubMed abstracts and full-text PubMed Central articles for marker names.</Description>
        <DbTo>unists</DbTo>
      </Link>
    </LinkList>
  </DbInfo>
</eInfoResult>
\end{lstlisting}
\end{frame}



%%%%%%%%%%%%%%%%%%%%%%%%%%%%%%%%%%%%%%%%%%%
%%
%% ESearch
%%
%%%%%%%%%%%%%%%%%%%%%%%%%%%%%%%%%%%%%%%%%%%

\hugeslide{ESearch}

\begin{frame}[fragile]
\frametitle{ESearch}
\begin{itemize}
\item Provides a list of UIDs matching a text query
\item Posts the results of a search on the History server
\item Downloads all UIDs from a dataset stored on the History server
\item Combines or limits UID datasets stored on the History server
\item Sorts sets of UIDs
\end{itemize}
\end{frame}



\end{document}

