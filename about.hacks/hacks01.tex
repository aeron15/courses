\documentclass[12pt]{article}
\usepackage[utf8]{inputenc}
\usepackage{hyperref}
\usepackage{graphicx}
\usepackage{listings}
\usepackage{amssymb}
\usepackage{color}
\usepackage{xcolor}
\usepackage{indentfirst}
\usepackage{makeidx}

\makeindex

\title{Programming Hacks}
\author{Pierre Lindenbaum PhD.}
\date{\today}

\lstset{language=bash,frame=single,backgroundcolor=\color{lightgray},numbers=left,breaklines=true,breakautoindent=true,basicstyle=\small}

\begin{document}
\maketitle

\tableofcontents

\section{C/C++}
\subsection{Scanning a directory tree}
\url{http://stackoverflow.com/questions/3833581}
\begin{lstlisting}
static int callback(const char *fpath, const struct stat *sb, int typeflag) {
	{
	(...)
	}
ftw(".", callback, 16);
\end{lstlisting}

\section{Make}
\subsection{Patterns}
\begin{lstlisting}
%.pdf:%.dvi
	dvipdf $<
%.dvi:%.tex
	latex $<
\end{lstlisting}
\subsection{Phony targets}
\url{http://www.gnu.org/s/hello/manual/make/Phony-Targets.html}
'.PHONY' declares the target as "symbolic". e.g. "clean". It will be executed even if there is a file named "clean".
\begin{lstlisting}
.PHONY : clean
clean:
             rm *.o temp
\end{lstlisting}

\section{Java}
\subsection{Double Brace Initialization}
\url{http://www.c2.com/cgi/wiki?DoubleBraceInitialization}
\begin{lstlisting}
removeProductsWithCodeIn(new HashSet<String>() {{
   add("XZ13s");
   add("AB21/X");
   add("YYLEX");
   add("AR5E");
 }});
\end{lstlisting}
\subsection{Joint union in type parameter variance}
\begin{lstlisting}
public class Baz<T extends Foo & Bar> {}
\end{lstlisting}
\section{C/C++}
\subsection{List predefined macros in gcc}
\url{http://en.wikipedia.org/wiki/C_preprocessor#Compiler-specific_predefined_macros}
\begin{lstlisting}
gcc -dM -E - < /dev/null
\end{lstlisting}

\section{Unix}
\subsection{Regex with one column}
\begin{lstlisting}
$ echo -e "A B C\nA T G C\n" | awk -F ' ' '($2 ~ "B" )'
A B C
\end{lstlisting}
\subsection{Replace one column}
\begin{lstlisting}
$ echo -e "A B C\nA T G C\n" | awk -F ' ' '($3 = "B" )'
A B B
A T B C
  B

\end{lstlisting}
\subsection{csplit }
split a file into sections determined by context lines
\subsection{fold: split lines}
\begin{lstlisting}
$ echo "AAAAAAAAAAAAAAAAAA" | fold -w 3
AAA
AAA
AAA
AAA
AAA
AAA
\end{lstlisting}

\subsection{mv more than one file}
\begin{lstlisting}
mv path/{*.tex,*.cpp} WS/
\end{lstlisting}

\subsection{speed up sort}
\begin{lstlisting}
export LC_ALL=C
\end{lstlisting}

\subsection{Adding message when logging a server}
modify: /etc/motd
\subsection{finding executables}
\begin{lstlisting}
 find ./ -type f -perm -o+rx
\end{lstlisting}

\subsection{Merge PDF}
\begin{lstlisting}
gs -dNOPAUSE -sDEVICE=pdfwrite -sOUTPUTFILE=jeter.pdf -dBATCH status.*.pdf
\end{lstlisting}

\subsection{passwordless connection with ssh}
on client side:
\begin{lstlisting}
$ ssh-keygen -t rsa
Enter file in which to save the key (~/.ssh/id_rsa): 
/home/lindenb/.ssh/id_rsa already exists.
Overwrite (y/n)? y
Enter passphrase (empty for no passphrase): 
Enter same passphrase again: 
Your identification has been saved in ~/.ssh/id_rsa.
Your public key has been saved in ~/.ssh/id_rsa.pub.
The key fingerprint is:
xxx lindenb@hardyweinberg

##tell ssh-agent
$ ssh-add

## transfert your key

cat .ssh/id_rsa.pub | ssh LOGIN@HOST 'cat >> .ssh/authorized_key'

\end{lstlisting}
on server side, if needed
\begin{lstlisting}
$ chmod 700 .ssh
$ chmod 600 .ssh/authorized_keys
\end{lstlisting}

\section{Virtual ssh directory}
\begin{lstlisting}
sudo apt-get install sshfs
mkdir VIRTUAL
sshfs  LOGIN@HOST:/data/users/me VIRTUAL
\end{lstlisting}

\section{HTML/Mozilla}
\subsection{multiple file input in Firefox 3.6}
\url{http://hacks.mozilla.org/2009/12/multiple-file-input-in-firefox-3-6/}
\begin{lstlisting}
<input type="file" multiple=""/>
\end{lstlisting}
\subsection{Mozilla JS shell}
\begin{lstlisting}
 xpcshell-2.0 -e 'print("Hello");'
\end{lstlisting}
\section{Postcript}
Including a picture: use the run operator.
\begin{lstlisting}
%!PS
0 0 moveto
100 100 lineto
stroke
(jeter.eps) run
0 0 moveto
100 100 lineto
stroke
\end{lstlisting}
\section{Github}
You've cloned a repo but you want to update the data from your colleague's original repository:
\begin{lstlisting}

$ git remote add parent "https://github.com/michaelbarton/bioinformatics-career-survey.git"
$ git pull parent master
\end{lstlisting}
\section{Javascript}
\paragraph{Mozilla/Xul interpreter:}
\begin{lstlisting}
export LD_LIBRARY_PATH=/usr/lib/xulrunner-2.0
JavaScript-C 1.8.0 pre-release 1 2007-10-03
usage: xpcshell [-g gredir] [-r manifest]... [-PsSwWxCij] [-v version] [-f scriptfile] [-e script] [scriptfile] [scriptarg...]
\end{lstlisting}
\section{Ecore/XMI}
\paragraph{Where is the XML schema for ECore}
%%\url{http://dev.eclipse.org/viewcvs/viewvc.cgi/org.eclipse.emf/org.eclipse.emf/plugins/org.eclipse.emf.ecore/model/Ecore.xsd?view=co\&root=Modeling_Project} .

\section{Apache2}
%allowing public_html:
%via: http://kimbriggs.com/computers/computer-notes/linux-notes/apache2-public_html-virtual-directories.file

\begin{lstlisting}

me@myhost$ cd /etc/apache2/mods-enabled
me@myhost$ sudo ln -s ../mods-available/userdir.conf userdir.conf
me@myhost$ sudo ln -s ../mods-available/userdir.load userdir.load
me@myhost$ sudo /etc/init.d/apache2 restart 
$ mkdir -p ~/public_html/cgi-bin
(...)
 chmod a+x flickr.cgi
 
edit:  sudo nano /etc/apache2/apache2.conf
and append:

<Directory /home/*/public_html/cgi-bin/>
Options ExecCGI
SetHandler cgi-script
</Directory>
\end{lstlisting}




\end{document}
