\documentclass{article}
\usepackage[utf8]{inputenc}
\usepackage{hyperref}
\usepackage{graphicx}
\usepackage{listings}
\usepackage{amssymb}
\usepackage{color}
\usepackage{indentfirst}
\newcommand{\menu}[1]{\colorbox{yellow}{\texttt{#1}}}

\def\emf{\textbf{EMF} }
\date{\today}
\title{EMF Notebook}
\author{Pierre Lindenbaum PhD\\\texttt{plindenbaum@yahoo.fr}}
\begin{document}
\maketitle

\begin{abstract}
\emf (The Eclipse Modeling Framework) is a Java framework and \textbf{code generation facility} for building tools and other applications based on a structured model.  Once you specify an EMF model, the \textbf{EMF generator} can create a corresponding set of Java implementation classes. 
\end{abstract}

\section{Generating the Model}
A model is created and defined in the \textbf{Ecore} format, which is basically a sub-set of UML Class diagrams.\\
\subsection{Create the Project}
\begin{itemize}
\item Open Eclipse
\item Menu \menu{File $\rightarrow$  New  $\rightarrow$ New Project... $\rightarrow$ Empty EMF Project}
\item Name the project \textbf{EMF03-Model}
\end{itemize}
\subsection{Create the ECore model}
create a new 'ECore' file in the model folder in your new modeling project.
\begin{itemize}
\item Menu \menu{File $\rightarrow$  New  $\rightarrow$ Other.. $\rightarrow$ ECore Model}
\item Wizard: select the folder \textbf{EMF03-Model/model}.
\item Wizard: set the filename as \textbf{lims.ecore}.
\item Wizard: set Model Object as  \textbf{EPackage}.
\item Menu \menu{Window  $\rightarrow$ Show View $\rightarrow$ Other $\rightarrow$ Properties}
\end{itemize}


\input{jeter01.tex}

\section{References}
\begin{itemize}
\item The Eclipse Modeling Framework Overview : \url{http://help.eclipse.org/juno/index.jsp?topic=%2Forg.eclipse.emf.doc%2Freferences%2Foverview%2FEMF.html}
\end{itemize}

\end{document}
