\documentclass{article}
\usepackage[utf8]{inputenc}
\usepackage{hyperref}
\usepackage{graphicx}
\usepackage{listings}
\usepackage{amssymb}
\usepackage{color}
\usepackage{xcolor}
\usepackage{indentfirst}
\usepackage{makeidx}
\addtolength{\oddsidemargin}{-.875in}
\addtolength{\evensidemargin}{-.875in}
\addtolength{\textwidth}{1.75in}

\makeindex

\newcommand{\ruby}{\huge{Ruby}}

\newcommand{\csection}{\huge{C}.\\}

\lstset{language=ruby,frame=single,backgroundcolor=\color{yellow},numbers=left,breaklines=true,breakautoindent=true,basicstyle=\small}
\date{\today}
\title{Ruby, my notebook.\\\includegraphics[scale=1.0,width=64px,height=64px]{ruby64.png}}
\author{Pierre Lindenbaum\\\href{https://twitter.com/yokofakun}{@yokofakun}\\\url{http://plindenbaum.blogspot.com} }



\begin{document}
\maketitle

\tableofcontents

\section{General}
\subsection{NULL}
\begin{lstlisting}
nil
\end{lstlisting}

\subsection{Global variable}
\begin{lstlisting}
$var1 = "Hello"
\end{lstlisting}
\subsection{Instance variable}
\begin{lstlisting}
@var2 = "World"
\end{lstlisting}

\subsection{Class variable}
\begin{lstlisting}
@@var3 = "blabala"
\end{lstlisting}

\subsection{Integer}
\subsubsection{times}
\begin{lstlisting}
10.times { print "CA" }
=>CACACACACACACACACACA
\end{lstlisting}
\subsubsection{upto}
\begin{lstlisting}
5.upto(7) {|i| puts "i=#{i};"}
i=5;
i=6;
i=7;
\end{lstlisting}

\subsection{String shortcuts}
\begin{lstlisting}
$var1 = "Hello"
puts "#$var1, World!"
\end{lstlisting}

\subsection{Function}
\begin{lstlisting}
def say_hello(name)
	result = "Hello, " + name
	return result
end
puts say_hello("World")
\end{lstlisting}
or
\begin{lstlisting}
$var1 = "Hello"
def say_hello(name)
	result = "#$var1, #{name}"
	return result
end
puts say_hello("World")
\end{lstlisting}

\subsection{Arrays}

\begin{lstlisting}
a = [ TRUE, 'GAATTC', "GGATCC", 0.5]
a[0]
=> true
> puts "EcorRI cuts #{a[1]}"
=> EcorRI cuts GAATTC
a[3] = nil
a
=> [true, "GAATTC", "GGATCC", nil]
\end{lstlisting}

or using '\%w'
\begin{lstlisting}
 %w{ EcoRI BamHI PstI NotchI }
=> ["EcoRI", "BamHI", "PstI", "NotI"]
\end{lstlisting}

\subsection{Associative Arrays}
\begin{lstlisting}
> map={ 'EcoRI' => 'GAATTC', 'BamHI' => 'GGATACC', 'PstI' => 'CGATCG' }
=> {"EcoRI"=>"GAATTC", "BamHI"=>"GGATACC", "PstI"=>"CGATCG"}

> puts map["EcoRI"]
=> GAATTC

> puts "Enzyme is #{map['EcoRI']}"
=> Enzyme is GAATTC
\end{lstlisting}

or

\begin{lstlisting}
> m=Hash.new()
=> {}
> m['EcoRI']="GAATTC";
> puts m['BamHI']
=> nil
> puts m['EcoRI']
GAATTC
\end{lstlisting}


\subsection{If-Then-Else}
\begin{lstlisting}
enz="EcoRI"

if enz=="BamHI"
	print "GGATCC"
elsif enz=="EcoRI"
	print "GAATTC"
else
	puts "Another enzyme"
end

=> GAATTC=> nil
\end{lstlisting}

\subsection{While loop}
\begin{lstlisting}
seq="";
while seq.length < 100
	seq += "A"
end
print seq
AAAAAAAAAAAAAAAAAAAAAAAAAAAAAAAAAAAAAAAAAAAAAAAAAAAAAAAAAAAAAAAAAAAAAAAAAAAAAAAAAAAAAAAAAAAAAAAAAAAA=> nil
\end{lstlisting}
or
\begin{lstlisting}
seq=""
seq += "A"   while seq.length < 100
puts seq
AAAAAAAAAAAAAAAAAAAAAAAAAAAAAAAAAAAAAAAAAAAAAAAAAAAAAAAAAAAAAAAAAAAAAAAAAAAAAAAAAAAAAAAAAAAAAAAAAAAA
=> nil
\end{lstlisting}

\subsection{Regex}
test regex
\begin{lstlisting}
s="ATGACTACGATC"

if s =~ /[ATGC]+/
	puts "Acid Nucleic"
else
	puts "Not a DNA"
end

=> Acid Nucleic
\end{lstlisting}

replace first occurence of regex:
\begin{lstlisting}
puts "GAATCGTACGATCGCTAGA".sub(/T/,"U")
=> GAAUCGTACGATCGCTAGA
\end{lstlisting}
replace all occurences of regex:
\begin{lstlisting}
puts "GAATCGTACGATCGCTAGA".gsub(/T/,"U")
=> GAAUCGUACGAUCGCUAGA
\end{lstlisting}
\subsection{Yield statement}

\begin{lstlisting}
def print_seq
	puts "<Sequence>"
	yield
	puts "</Sequence>"
end
print_seq { puts "GAAATTTATAC" }

=> <Sequence>
GAAATTTATAC
</Sequence>
\end{lstlisting}

\begin{lstlisting}
def print_enz
	puts "<enzymes>"
	yield("EcoRI","GAATTC")
	yield("BamHI","GGATCC")
	puts "</enzymes>"
end
print_enz {|name,site| puts "<site name=\"#{name}\">#{site}</site>" }
print_enz {|name,site| puts "<enzyme><name>#{name}</name><site>#{site}</site></enzyme>" }

=> <enzymes>
<site name="EcoRI">GAATTC</site>
<site name="BamHI">GGATCC</site>
</enzymes>

<enzymes>
<enzyme><name>EcoRI</name><site>GAATTC</site></enzyme>
<enzyme><name>BamHI</name><site>GGATCC</site></enzyme>
</enzymes>
\end{lstlisting}

\subsubsection{ 'each'}
\begin{lstlisting}
%w( EcorRI  BamHI PstI NotI).each {|name| puts name }
=>
EcorRI
BamHI
PstI
NotI
\end{lstlisting}

\subsection{IO}
\subsubsection{Reading}
\begin{lstlisting}
line=gets
puts line
\end{lstlisting}

\subsection{Class}
\begin{lstlisting}
class GeneticCode
	def initialize(name,ncbiString)
		@name=name
		@ncbiString=ncbiString
	end
end

std = GeneticCode.new(
	"standard",
	"FFLLSSSSYY**CC*WLLLLPPPPHHQQRRRRIIIMTTTTNNKKSSRRVVVVAAAADDEEGGGG"
	)
\end{lstlisting}
\subsubsection{inspect}
\begin{lstlisting}
std = GeneticCode.new(
	"standard",
	"FFLLSSSSYY**CC*WLLLLPPPPHHQQRRRRIIIMTTTTNNKKSSRRVVVVAAAADDEEGGGG"
	)

puts std.inspect
=>#<GeneticCode:0xb77b0a04 @ncbiString="FFLLSSSSYY**CC*WLLLLPPPPHHQQRRRRIIIMTTTTNNKKSSRRVVVVAAAADDEEGGGG", @name="standard">
\end{lstlisting}


\subsubsection{to\_s}
\begin{lstlisting}
class GeneticCode
	def initialize(name,ncbiString)
		@name=name
		@ncbiString=ncbiString
	end
	
	def to_s
		"Genetic code #@name"
	end
end

std = GeneticCode.new(
	"standard",
	"FFLLSSSSYY**CC*WLLLLPPPPHHQQRRRRIIIMTTTTNNKKSSRRVVVVAAAADDEEGGGG"
	)

puts std

=> "Genetic code standard"
\end{lstlisting}



\end{document}
